\section{Discussion}
\label{sec:discussion}

Figure~\ref{fig:hinton} is the Hinton diagram showing the relationship
between the most frequent tags and clusters found by the collapsed
algorithm (\collapseResult\% many-to-one accuracy).  In this section
we present a qualitative comparison of gold standard tags and
discovered clusters.

\paragraph{Nouns and adjectives:} Most nouns ({\sc nn*}) are split between the
clusters represented by the first seven columns of the Hinton graph,
but not in the way Penn Treebank splits them.  For example cluster 27
brings together titles like {\em Mr.}, {\em Mrs.}, {\em Dr.}
etc. which does not exist as a separate class in the gold tags.
Cluster 29 is the largest adjective ({\sc jj}) cluster, however it
also has noun members probably due to the difficulty of separating
noun-noun compounds and adjective modification.

\paragraph{Verbs and adverbs:}  Clusters 9 and 33 contain general
verbs ({\sc vb*}), but the verbs ``be'' (26), ``say'' (21), and
``have'' (34) have been split into their own clusters indicated in
parantheses, presumably because they are not generally substitutable
with the rest.  Adverb ({\sc rb}) is an amorphous class and the
algorithm seems to have difficulty isolating it in a cluster.

\paragraph{Determiners and prepositions:}  We see a fairly clean
separation of determiners ({\sc dt}) and prepositions ({\sc in}) from
other parts of speech, although each has been subdivided into further
groups by the algorithm.  For example cluster 39 contains general
prepositions but ``of'' (43), ``in'' (13), and ``for'' (3) are split
into their own clusters.  Determiners ``the'' (8), ``a'' (12), and
capitalized ``The''/''A'' (6) are also split into their own clusters.

\paragraph{Closed-class items:}  Most closed-class items are cleanly
separated into their own clusters as seen in the lower right hand
corner of the diagram.
