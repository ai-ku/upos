\section{Algorithm}

In this section we describe the components of our algorithm and their
relationship with each other.  The algorithm predicts the syntactic
category of a word in a given context based on its substitutes.  In
other words first we construct the co-occurrence representation of
words and their substitutes with the help of a language model and then
map each value in the co-occurrence data to a corresponding embedding
on a $n$ dimensional sphere using the S-CODE algorithm.  Finally, we
apply k-means clustering to categorize the word embeddings by which we
induced the word categories.  In the next subsection we detail the
construction of co-occurrence representation of the input, in
Subsection~\ref{??} we explain the embedding calculation and finally
in Subsection~\ref{??} we describe the different ways of embedding
clustering.\todo{do we need subsections?}

\subsection{Co-occurences}

% How we represent the context?
% How we relate the word and the context?

Word contexts are represented by substitutes sampled from the
corresponding substitute word distributions.  Substitute words are
sampled with replacement according to the substitute distributions
that are calculated based on an n-gram language model.  The sample set
of the substitute word distribution is the vocabulary of the language
model.\footnote{Sampled substitutes might include the unknown word tag
  ``unk'' since it is in the language model vocabulary.  For example
  substitutes of proper nouns usually include ``unk'' as a
  substitute.}  To capture the relation between each word and their
contexts we construct a co-occurrence representation by pairing the
words and their substitutes.  Table~\ref{tab:samples} shows sampled
substitutes of each word and their co-occurrence representation on an
example sentence.

\begin{table}[h]
\caption{The table on the left shows possible three substitutes
  sampled for each of the positions in the example sentence
  \textit{``Pierre Vinken, 61 years old, will join the board as a
    nonexecutive director Nov.~29 .''} based on a 4-gram language
  model and the one on the right represents the input sentence as a
  co-occurrences of words and their substitutes.  To distinguish
  between words and their substitutes we concatenate ``w'' and ``s'',
  respectively, on the co-occurrence data.}
\begin{tabular}{|ll|} \hline
\textbf{Word} & \textbf{Sampled Substitutes}\\
\hline
\textbf{Pierre} & \textit{Mr.}  \textit{Pierre}  \textit{Mr.}\\
\textbf{Vinken} & \textit{unk}  \textit{Beregovoy}  \textit{Cardin}\\
\textbf{,} & \textit{,}  \textit{,}  \textit{,}\\
\textbf{61} & \textit{48}  \textit{52}  \textit{41}\\
\textbf{years} & \textit{years}  \textit{years}  \textit{years}\\
\textbf{old} & \textit{old}  \textit{old}  \textit{old}\\
\textbf{,} & \textit{,}  \textit{,}  \textit{,}\\
\textbf{will} & \textit{will}  \textit{will}  \textit{will}\\
\textbf{join} & \textit{head}  \textit{join}  \textit{leave}\\
\textbf{the}  & \textit{its}  \textit{its}  \textit{the}\\
\textbf{board} & \textit{board}  \textit{company}  \textit{firm}\\
\textbf{as} & \textit{as}  \textit{as}  \textit{as}\\
\textbf{a} & \textit{a}  \textit{a}  \textit{a}\\
\textbf{nonexecutive} & \textit{nonexecutive}  \textit{non-executive}  \textit{nonexecutive}\\
\textbf{director} & \textit{chairman}  \textit{chairman}  \textit{director}\\
\textbf{Nov.} & \textit{April}  \textit{May}  \textit{of}\\
\textbf{29} & \textit{16}  \textit{29}  \textit{9}\\
\textbf{.} & \textit{.}  \textit{.}  \textit{.}\\
\hline
\end{tabular}
\quad
\begin{tabular}{|ll|}
\hline
\textbf{Word} & \textbf{Substitute}\\
\hline
\textbf{w:Pierre} & \textit{s:Mr.}\\
\textbf{w:Pierre} & \textit{s:Pierre}\\
\textbf{w:Pierre} & \textit{s:Mr.}\\
\textbf{w:Vinken} & \textit{s:unk}\\
\textbf{w:Vinken} & \textit{s:Beregovoy}\\
\textbf{w:Vinken} & \textit{s:Cardin}\\
%% \textbf{,} & \textit{,}\\
%% \textbf{,} & \textit{,}\\
%% \textbf{,} & \textit{,}\\
%% \textbf{61} & \textit{48}\\
%% \textbf{61} & \textit{52}\\
%% \textbf{61} & \textit{41}\\
%% \textbf{years} & \textit{years}\\
%% \textbf{years} & \textit{years}\\
%% \textbf{years} & \textit{years}\\
%% \textbf{old} & \textit{old}\\
%% \textbf{old} & \textit{old}\\
%% \textbf{old} & \textit{old}\\
%% \textbf{,} & \textit{,}\\
%% \textbf{,} & \textit{,}\\
%% \textbf{,} & \textit{,}\\
%% \textbf{will} & \textit{will}\\
%% \textbf{will} & \textit{will}\\
%% \textbf{will} & \textit{will}\\
\hdots&\\
\textbf{w:join} & \textit{s:head}\\
\textbf{w:join} & \textit{s:join}\\
\textbf{w:join} & \textit{s:leave}\\
\textbf{w:the} & \textit{s:its}\\
\textbf{w:the} & \textit{s:its}\\
\textbf{w:the} & \textit{s:the}\\
\hdots&\\
%% \textbf{board} & \textit{board}\\
%% \textbf{board} & \textit{company}\\
%% \textbf{board} & \textit{firm}\\
%% \textbf{as} & \textit{as}\\
%% \textbf{as} & \textit{as}\\
%% \textbf{as} & \textit{as}\\
%% \textbf{a} & \textit{a}\\
%% \textbf{a} & \textit{a}\\
%% \textbf{a} & \textit{a}\\
%% \textbf{nonexecutive} & \textit{nonexecutive}\\
%% \textbf{nonexecutive} & \textit{non-executive}\\
%% \textbf{nonexecutive} & \textit{nonexecutive}\\
\textbf{w:director} & \textit{s:chairman}\\
\textbf{w:director} & \textit{s:chairman}\\
\textbf{w:director} & \textit{s:director}\\
\hdots&\\
%% \textbf{Nov.} & \textit{April}\\
%% \textbf{Nov.} & \textit{May}\\
%% \textbf{Nov.} & \textit{of}\\
%% \textbf{29} & \textit{16}\\
%% \textbf{29} & \textit{29}\\
%% \textbf{29} & \textit{9}\\
%% \textbf{.} & \textit{.}\\
%% \textbf{.} & \textit{.}\\
%% \textbf{.} & \textit{.}\\
\hline
\end{tabular}
\label{tab:samples}
\end{table}
