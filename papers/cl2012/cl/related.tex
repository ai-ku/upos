\section{Related Work}
\label{sec:related}
%%% This part looks fine
There are several good reviews of algorithms for unsupervised
part of speech induction
\cite{Christodoulopoulos:2010:TDU:1870658.1870714,Gao:2008:CBE:1613715.1613761}
and models of syntactic category acquisition \cite{ambridge2011child}.

This work is to be distinguished from supervised part of speech
disambiguation systems, which use labeled training data
\cite{Toutanova:2003:FPT:1073445.1073478}, unsupervised disambiguation
systems, which use a dictionary of possible tags for each word
\cite{yatbaz-yuret:2010:POSTERS}, or prototype driven systems
which use a small set of prototypes for each class
\cite{Haghighi:2006:PLS:1220835.1220876}.  The problem of induction is
important for studying under-resourced languages that lack labeled
corpora and high quality dictionaries.  It is also essential in
modeling child language acquisition because every child manages to
induce syntactic categories without access to labeled sentences,
labeled prototypes, or dictionary constraints.
%%%%
%\paragraph{Distributional vs. Feature Based}
% \subsection{Distributional vs. Feature Based}

Models of unsupervised part of speech induction fall into two broad
groups based on the information they utilize.  Distributional models
only use word types and their context statistics.  Word-feature models
incorporate additional morphological and orthographic features.

\subsection{Distributional models}

Distributional models can be further categorized into three subgroups
based on the learning algorithm.  The first subgroup represents each
word type/token with its context vector and clusters these vectors
accordingly \cite{Schutze:1995:DPT:976973.976994}.  Work in modeling
child syntactic category acquisition has generally followed this
clustering approach
\cite{redington1998distributional,mintz2003frequent}.  The second
subgroup consists of probabilistic models based on the Hidden Markov
Model (HMM) framework \cite{Brown:1992:CNG:176313.176316}.  A third
group of algorithms constructs a low dimensional representation of the
data that represents the empirical co-occurrence statistics of word
types \cite{globerson2007euclidean}, which is covered in more detail
in Section~\ref{app:codethr}.

\paragraph{Clustering}
Clustering based methods represent the context using the neighboring
words, typically a single word on the left and a single word on the
right called a ``frame'' (e.g., {\em {\bf the} dog {\bf is}; {\bf the}
  cat {\bf is}}).  They cluster word types rather than word tokens
based on the frames they occupy thus employing one-tag-per-word
assumption from the beginning (with the exception of
\cite{mintz2003frequent,20674613} and some methods in
\cite{Schutze:1995:DPT:976973.976994}).  They may suffer from the data
sparsity caused by the infrequent words and the infrequent contexts.
The solutions suggested either restrict the set of words and set of
contexts to be clustered to the most frequently observed, or use
dimensionality reduction.  Redington et
al. \shortcite{redington1998distributional} define context similarity
based on the number of common frames bypassing the data sparsity
problem but achieve lower scores than the best performing systems.
Mintz \shortcite{mintz2003frequent} only uses the most frequent 45
frames to cluster tokens and achieves 98\% unsupervised accuracy on
the tokens observed in the most frequent 45 frames.  Similar to
Mintz's work, St Clair et al. \shortcite{20674613} show that systems
that model the left and right frames of tokens seperately perform
better than the frequent frames both interms of token clustering
accuracy and the token coverage.  Biemann
\shortcite{biemann2006unsupervised} contructs a graph based view of
the most frequent 10,000 words using contexts formed from the most
frequent 150-200 words and clusters the tokens.  Sch\"utze
\shortcite{Schutze:1995:DPT:976973.976994} and Lamar et
al. \shortcite{lamar-EtAl:2010:Short} employ SVD to enhance similarity
between less frequently observed word types and contexts.  Lamar et
al. \shortcite{Lamar:2010:LCU:1870658.1870736} represent each context
by the currently assigned left and right tag (which eliminates data
sparsity) and cluster word types using a soft k-means style iterative
algorithm.  They report the best clustering result to date of .708
many-to-one accuracy on the PTB.
% all except schutze cluster word types

\paragraph{HMMs}
The prototypical bitag HMM model maximizes the likelihood of the
corpus $w_1 \ldots w_n$ expressed as $P(w_1|c_1)\prod_{i=2}^n
P(w_i|c_i) P(c_i|c_{i-1})$ where $w_i$ are the word tokens and $c_i$
are their (hidden) tags.  One problem with such a model is its
tendency to distribute probabilities equally and the resulting
inability to model highly skewed word-tag distributions observed in
hand-labeled data \cite{johnson:2007:EMNLP-CoNLL2007}.  To favor
sparse word-tag distributions one can enforce a strict
one-tag-per-word solution (type clustering)
\cite{Brown:1992:CNG:176313.176316,Clark:2003:CDM:1067807.1067817},
use sparse priors in a Bayesian setting
\cite{goldwater-griffiths:2007:ACLMain,johnson:2007:EMNLP-CoNLL2007},
or use posterior regularization
\cite{Ganchev:2010:PRS:1859890.1859918}.  Each of these techniques
provide significant improvements over the standard HMM model: for
example Gao and Johnson \shortcite{Gao:2008:CBE:1613715.1613761} show
that sparse priors can gain from 4\% (.62 to .66 on the PTB) in
cross-validated many-to-one accuracy.  However Christodoulopoulos et
al. \shortcite{Christodoulopoulos:2010:TDU:1870658.1870714} show that
the older one-tag-per-word models such as
\cite{Brown:1992:CNG:176313.176316} outperform the more sophisticated
sparse prior and posterior regularization methods both in speed and
accuracy (the Brown model gets .68 many-to-one accuracy on the PTB).
Given that 93.69\% of the word occurrences in human labeled data are
tagged with their most frequent part of speech
\cite{Toutanova:2003:FPT:1073445.1073478}, this is probably not
surprising; one-tag-per-word is a fairly good first approximation for
induction.

%% \paragraph{Co-occurrence embedding}
%% Embedding algorithms maps the original data to a lower dimensional
%% space that preserves certain similarity structure defined between the
%% objects in the data.  However in real life problems such as POS
%% induction the similarity between contexts and word types can not be
%% trivially constructed given that only the co-occurrence information
%% exists between word types and their contexts.
%% S-Code\cite{NIPS2010_1196} which is a spherical extension of
%% \cite{Globerson:2007:EEC:1314498.1314572} algorithm maps the
%% word-types and their contexts on to a low dimensional Euclidean sphere
%% while preserving the co-occurrence statistics between them and then
%% clusters the low dimensional data with k-means algorithm.  They report
%% .688 many-to-one accuracy on PTB with 45 tags.

\subsection{Word-feature models}
One problem with the algorithms in the previous section is the poverty
of their input features.  Of the syntactic, semantic, and
morphological information linguists claim underlie syntactic
categories, context vectors or bitag HMMs only represent limited
syntactic information in their input.  Experiments incorporating
morphological and orthographic features into HMM based models
demonstrate significant improvements.
\cite{Clark:2003:CDM:1067807.1067817,bergkirkpatrick-klein:2010:ACL,blunsom-cohn:2011:ACL-HLT2011}
incorporate similar orthographic features and report improvements of
3, 7, and 10\% respectively over the baseline Brown model.

\cite{Clark:2003:CDM:1067807.1067817,blunsom-cohn:2011:ACL-HLT2011}
cluster types by incorporating similar orthographic features and
report improvements of 3 and 10\% respectively over the baseline Brown
model.  Berg-Kirkpatrick et al. incorporate orthographic features into
EM algorithm where they replace the multinomial components with
miniature logistic regressions and cluster tokens while improving the
Brown model by 7\%.

Christodoulopoulos et
al. \shortcite{Christodoulopoulos:2010:TDU:1870658.1870714} use
prototype based features as described in
\cite{Haghighi:2006:PLS:1220835.1220876} with automatically induced
prototypes and report an 8\% improvement over the baseline Brown model
by clustering tokens.  Abend et
al. \shortcite{Abend:2010:IUP:1858681.1858813} train a
prototype-driven model with morphological features by first clustering
the high frequent types as the landmarks and then assigning the
remaining types to these landmark clusters.  Christodoulopoulos et
al. \shortcite{christodoulopoulos-goldwater-steedman:2011:EMNLP}
define a type-based Bayesian multinomial mixture model in which each
word instance is generated from the corresponding word type mixture
component and word contexts are represented as features.  They achieve
a .728\ \mto\ score by extending their model with additional
morphological and alignment features gathered from parallel corpora.
To our knowledge, nobody has yet tried to incorporate phonological or
prosodic features in a computational model for syntactic category
acquisition.

\subsection{Paradigmatic representations}

Yatbaz et al. \shortcite{yatbaz-sert-yuret:2012:EMNLP-CoNLL} explore
the paradigmatic representation of the word contexts by modeling the
co-occurrence of words and their substitutes within the CODE
framework.  Their experiments on the PTB types shows that paradigmatic
representation improves the state-of-the-art \mto\ and V-measure (\vm)
accuracies of both distributional models and models with additional
word features.  This paper builds on that preliminary work by
%% (1) exploring clustering of substitute vectors,
(1) exploring induction of part-of-speechs at token level (in addition
to type level), (2) improving the model for using additional features,
and (3) experimenting with additional languages.


%% This paper defines a new representation that models the features from
%% different domains as separate variables and relates them to each other
%% by using their co-occurrence statistics with a shared variable
%% (i.e. word types).

\subsection{Evaluation}
%% It is natural to question the merit of evaluating unsupervised results
%% by comparing them to gold standard tags.  For example
%% \cite{freudenthal2005resolution} argue that a verb category (such as
%% {\sc vb} in Penn Treebank) that contains verbs that can and cannot be
%% used in certain constructions (e.g. the imperative) and verbs that can
%% be used as both auxiliaries and main verbs (e.g., {\em do, have}) does
%% not in fact constitute a set of items that could be substituted for
%% one another in particular sentences.  Such a category fails the
%% standard linguistic definition of a syntactic category and children do
%% not seem to make errors of substituting such words in utterances
%% (e.g. {\em''What do you want?''} vs. {\em *''What put you want?''}).
%% They suggest evaluating models by incorporating a production component
%% that allows the model's output to be compared to speech produced by
%% children exposed to the same input.  \cite{frank2009evaluating},
%% motivated by the lack of gold standards for many novel languages,
%% suggest comparing a system's clusters to a set of clusters created
%% from {\em substitutable frames}.  They create frames using two words
%% appearing in the corpus with exactly one word between and calculate
%% precision, recall, and F-score of the system's clustering.
%% Statistical parsers or factored machine learning systems could also be
%% sources of extrinsic evaluation for induced syntactic categories.  We
%% hope such extrinsic evaluation will be more widespread in the future
%% but nevertheless use many-to-one (\mto) and V-measure (\vm) metric on
%% the 45-tag Penn Treebank gold standard to evaluate the current work.

We report many-to-one and V-measure scores for our experiments as
suggested in \cite{Christodoulopoulos:2010:TDU:1870658.1870714}.  The
many-to-one (\mto) evaluation maps each cluster to its most frequent
gold tag and reports the percentage of correctly tagged instances.
The \mto\ score naturally gets higher with increasing number of
clusters but it is an intuitive metric when comparing results with the
same number of clusters.  The V-measure (\vm) \cite{rosenberg2007v} is
an information theoretic metric that reports the harmonic mean of
homogeneity (each cluster should contain only instances of a single
class) and completeness (all instances of a class should be members of
the same cluster).  In Section~\ref{sec:discuss} we argue that
homogeneity is perhaps more important in part of speech induction and
suggest \mto\ with a fixed number of clusters as a more intuitive
metric.

%% \paragraph{Many-to-one (MTO)} metric measures the cluster purity 
%% therefore each cluster is mapped to the most frequent gold tag of the
%% words observed in that cluster.  \mto metric is the most intuitive
%% measure, however it has a bias towards to the increasing number of
%% clusters thus it should be used when the number of clusters is fixed
%% \cite{christodoulopoulos-goldwater-steedman:2011:EMNLP}.
%% \paragraph{V-measure (VM)} is an entropy based metric that calculates
%% weighted average of cluster homogeneity and completeness in a similar
%% fashion with F-measure recall and precision, respectively.  The
%% cluster homogeneity is an analogy of the cluster purity in \mto.
%% Although \vm is more robust to variable number of clusters, it is less
%% intuitive than \mto
%% \cite{christodoulopoulos-goldwater-steedman:2011:EMNLP}.  Moreover it
%% punishes the clustering when words with same gold tag distributed over
%% different clusters which contradicts with the idea of POS induction
%% since the aim is to cluster words according to the underlying natural
%% grouping.

%%% DY: where is the original paper that introduced V-measure?
%%% DY: are you sure Chris2011 is the right citation, shouldn't it be Chris2010?

