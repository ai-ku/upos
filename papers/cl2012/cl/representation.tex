\section{Representing Word Context}
\label{sec:representation}

% example substitute vectors (both syntactic and semantic)

The two examples below illustrate the advantage of paradigmatic
representations in uncovering similarities where no overt similarity
that can be captured by a syntagmatic representation exists.  The word
``board'' from the first sentence and the word ``council'' from the
second sentence have no common neighbors except the determiner
``the''.  The paradigmatic representation captures the similarity of
these words by suggesting the same top substitutes for both (the
numbers in parentheses give substitute probabilities):
\begin{quote}
 (1) \noindent{\em ``Pierre Vinken, 61 years old, will join the {\bf board} as a nonexecutive director Nov.~29.''}\\
% {\bf the:} its (.9011), the (.0981), a (.0006), $\ldots$\\
 {\bf board:} board (.4288), company (.2584), firm (.2024), bank (.0731), $\ldots$
\end{quote}

\begin{quote}
%  (2) \noindent {\em ``Blacks and Hispanics currently make up 38 \% of the city 's population and hold only 25 \% of the seats on the {\bf council} .''}\\
  (2) \noindent {\em ``$\ldots$ and hold only 25 \% of the seats on the {\bf council} .''}\\
 \noindent {\bf council:} board (.6591), company (.0795), firm (.0542), bank (.0154), $\ldots$
\end{quote}

The high probability substitutes reflect both semantic and syntactic
properties of the context.  Top substitutes for ``board'' and
``council'' are not only nouns, but specifically nouns compatible with
the semantic context.  Top substitutes for the word ``the'' in the
first example consist of words that can act as determiners: its
(.9011), the (.0981), a (.0006), $\ldots$

%% \subsection{Substitute Vectors}
%% \label{sec:subthr}
%% %%substitute vector applications

%% In this section we explore the usage of substitute vectors within the
%% context clustering framework by comparing similarity metrics,
%% dimensionality reduction techniques and clustering methods.  The first
%% subsection describes the computation of substitute vectors using a
%% statistical language model.  Section~\ref{sec:dist} gives a detailed
%% comparison of similarity metrics in the high dimensional substitute
%% vector space.  Section~\ref{sec:dimreduce} analyzes the application of
%% dimensionality reduction algorithms to substitute vectors.
%% Section~\ref{sec:clustering} presents a comparison of clustering
%% methods on the PTB.

\subsection{Computation of Substitute Vectors}
\label{sec:subcomp}

In this study, we predict the syntactic category of a word in a given
context based on its substitute vector.  The dimensions of the
substitute vector represent words in the vocabulary, and the entries
in the substitute vector represent the probability of those words
being used in the given context.  Note that the substitute vector is a
function of the context only and is indifferent to the target word.

%This section details the choice of the data set, the vocabulary and
%the estimation of substitute vector probabilities.

%% % what is the test data
%% The Wall Street Journal Section of the Penn Treebank \cite{treebank3}
%% was used as the test corpus (1,173,766 tokens, 49,206 types).
%% % what is the tag set
%% The treebank uses 45 part of speech tags which is the set we used as
%% the gold standard for comparison in our experiments.
%% % what is the LM training data
%% %Train => 5181717 126019973 690121813
%% To compute substitute probabilities we trained a language model using
%% approximately 126 million tokens of Wall Street Journal data
%% (1987-1994) extracted from CSR-III Text \cite{csr3text} (we excluded
%% the test corpus).
%% % how is the language model trained
%% We used SRILM \cite{Stolcke2002} to build a 4-gram language model with
%% Kneser-Ney discounting.
%% % what is the vocabulary
%% Words that were observed less than 20 times in the language model
%% training data were replaced by \textsc{unk} tags, which gave us a
%% vocabulary size of 78,498.
%% % perplexity
%% The perplexity of the 4-gram language model on the test corpus is 96.

% how are the substitutes computed
It is best to use both the left and the right context when estimating the
probabilities for potential lexical substitutes.  For example, in
\emph{``He lived in San Francisco suburbs.''}, the token \emph{San}
would be difficult to guess from the left context but it is almost
certain looking at the right context.  We define $c_w$ as the $2n-1$
word window centered around the target word position: $w_{-n+1} \ldots
w_0 \ldots w_{n-1}$ ($n=4$ is the n-gram order).  The probability of a
substitute word $w$ in a given context $c_w$ can be estimated as:
\begin{eqnarray}
  \label{eq:lm1}P(w_0 = w | c_w) & \propto & P(w_{-n+1}\ldots w_0\ldots w_{n-1})\\
  \label{eq:lm2}& = & P(w_{-n+1})P(w_{-n+2}|w_{-n+1})\nonumber\\
  &&\ldots P(w_{n-1}|w_{-n+1}^{n-2})\\
  \label{eq:lm3}& \approx & P(w_0| w_{-n+1}^{-1})P(w_{1}|w_{-n+2}^0)\nonumber\\
  &&\ldots P(w_{n-1}|w_0^{n-2})
\end{eqnarray}
where $w_i^j$ represents the sequence of words $w_i w_{i+1} \ldots
w_{j}$.  In Equation \ref{eq:lm1}, $P(w|c_w)$ is proportional to
$P(w_{-n+1}\ldots w_0 \ldots w_{n+1})$ because the words of the
context are fixed.  Terms without $w_0$ are identical for each
substitute in Equation \ref{eq:lm2} therefore they have been dropped
in Equation \ref{eq:lm3}.  Finally, because of the Markov property of
n-gram language model, only the closest $n-1$ words are used in the
experiments.

Near the sentence boundaries the appropriate terms were truncated in
Equation \ref{eq:lm3}.  Specifically, at the beginning of the sentence
shorter n-gram contexts were used and at the end of the sentence terms
beyond the end-of-sentence token were dropped.  

%% Rest of this section details the choice of the data set, the
%% vocabulary and the estimation of substitute probabilities.
%% For computational efficiency only the top 100 substitutes and their
%% unnormalized probabilities were computed for each of the 1,173,766
%% positions in the test set\footnote{The substitutes with unnormalized
%%   log probabilities can be downloaded from
%%   \mbox{\url{http://goo.gl/jzKH0}}.  For a description of the {\sc
%%     fastsubs} algorithm used to generate the substitutes please see
%%   \mbox{\url{http://arxiv.org/abs/1205.5407v1}}.  {\sc fastsubs}
%%   accomplishes this task in about 5 hours, a naive algorithm that
%%   looks at the whole vocabulary would take more than 6 days on a
%%   typical 2012 workstation.}.  The probability vectors for each
%% position were normalized to add up to 1.0 giving us the final
%% substitute vectors used in the rest of this study.

% what is the LM training data
%Train => 5181717 126019973 690121813

% what is the test data
The Wall Street Journal Section of the Penn Treebank \cite{treebank3}
was used as the test corpus (1,173,766 tokens, 49,206 types) to be
induced.
% what is the tag set
The treebank uses 45 part-of-speech tags which is the set we used as
the gold standard for comparison in our experiments.
% what is the LM training data
%Train => 5181717 126019973 690121813
To compute substitute probabilities we trained a language model using
approximately 126 million tokens of Wall Street Journal data
(1987-1994) extracted from CSR-III Text \cite{csr3text} (we excluded
the test corpus).
% how is the language model trained
We used SRILM \cite{Stolcke2002} to build a 4-gram language model with
Kneser-Ney discounting.
% what is the vocabulary
Words that were observed less than 20 times in the language model
training data were replaced by \textsc{unk} tags, which gave us a
vocabulary size of 78,498.
% perplexity
The perplexity of the 4-gram language model on the test corpus is 96.

%% To compute substitute probabilities we trained a language model using
%% approximately 126 million tokens of Wall Street Journal data
%% (1987-1994) extracted from CSR-III Text \cite{csr3text} (excluding
%% sections of the PTB).
%% % how is the language model trained
%% We used SRILM \cite{Stolcke2002} to build a 4-gram language model with
%% Kneser-Ney discounting.
%% % what is the vocabulary
%% Words that were observed less than 500 times in the LM training data
%% were replaced by \textsc{unk} tags, which gave us a vocabulary size of
%% 12,672.
%% % what is the test data
%% The first 24,020 tokens of the Penn Treebank Wall Street Journal
%% Section 00 (PTB24K) was used as the test corpus to be induced.  The corpus
%% size was kept small in order to efficiently compute full distance
%% matrices.  Substitution probabilities for 12,672 vocabulary words were
%% computed at each of the 24,020 positions.
%% % perplexity
%% The perplexity of the 4-gram language model on the test corpus was
%% 55.4 which is quite low due to using a small
%% vocabulary and in-domain data.
%% % what is the tag set
%% The treebank uses 45 part of speech tags which is the set we used as
%% the gold standard for comparison in our experiments.

