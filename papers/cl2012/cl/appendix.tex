\appendix
\section{Appendix A}
\label{app:lm}

\begin{table}[h]
%        \tiny   
  \begin{tabular}{l|l|c|c|c|c|c|c|c|}
    \cline{2-9}
    & & \multicolumn{3}{c|}{Language Model} & \multicolumn{4}{c|}{Test set}\\    \cline{2-9}
    & Language & Source & \specialcell{Sentence\\Count} & \specialcell{Word\\Count} & \specialcell{Sentence\\Count} & \specialcell{Word\\Count} & \specialcell{Perplexity\\(ppl)} & \specialcell{Unknown\\Word} \\ \cline{1-9}
    \multirow{1}{*}{\begin{sideways}\textbf{WSJ}\end{sideways}} 
    &English & News & 5,187,874 & 126,170,376 & 49,208 & 1,173,766 & ?? & ??\\
    & & & & && & &\\\hline
    \multirow{8}{*}{\begin{sideways}\textbf{MULTEXT-East}\end{sideways}}
    &Bulgarian& Wikipedia &1,596,399 & 32,511,616  & 6,682 & 101,173 & 655.202 & .0565\\
    &Czech & Wikipedia &3,059,678 & 59,698,049 & 6,752 & 100,368 & 1,069.67 & .0299\\
    &English & News & 5,187,874 & 126,170,376 & 6,737 & 118,424 & 265.246 & .0288\\
    &Estonian & Wikipedia &833,677 & 14,513,571 & 6,478 & 94,898 & 871.765 & .0654\\
    &Hungarian & Wikipedia &3,250,267& 66,069,788 & 6,768 & 98,426 & 742.676 & .0449\\
    &Romanian & Wikipedia &3,250,267&66,069,788  & 6,520 & 118,328 & 666.855 & .1074\\
    &Slovene & Wikipedia & 899,329&18,969,846 & 6,689 & 112,278 & 658.711 & .0389\\
    &Serbian & Wikipedia & 782,278 & 17,129,679 & 6,677 & 108,809 & 804.962 & .0580\\
    \hline % Conll06 data
    \multirow{10}{*}{\begin{sideways}\textbf{CoNLL06 Shared Task}\end{sideways}}
    &Bulgarian& Wikipedia &1,596,399 & 32,511,616  & 12,823 & 190,217 & 538.972 & .0043\\
    &Czech & Wikipedia &3,059,678 & 59,698,049 & 72,703 & 1,249,408 & 1,233.95 &.0250\\
    &Danish & Wikipedia &1,672,003 & 35,863,945 & 5,190 & 94,386 & 351.24 & .0393\\
    &Dutch & Wikipedia &8,266,922 & 159,978,524 & 13,349 & 195,069 & 390.818 & .0476\\
    &German & Wikipedia &22,454,543&437,777,863 & 39,216 & 699,610 & 680.036 & .0487\\
    &Portuguese & Wikipedia & 5,706,037 & 150,099,154 & 9071 & 206,678 & 378.656 & .0861\\
    &Slovene & Wikipedia & 899,329 & 18,969,846 & 1,534 & 28,750 & 663.053 & .0414\\
    &Spanish & Wikipedia &11,534,351 & 332,311,650& 3,306 & 89,334 & 274.418 & .0424\\
    &Swedish & Wikipedia &1,953,794 & 32,004,538& 11,042 & 191,467 & 1,233.95 & .0250\\
    &Turkish & Web &39,595,781 & 491,195,991& 4,997 & 47,605 & 868.829 & .0508\\
    \hline
  \end{tabular}
  \caption{Summary of language model training and test corpora
  realted information for each language in the test set.}
  \label{tab:lmstatistics}
\end{table}

\noindent Table~\ref{tab:lmstatistics} presents statistics related to the
language model training and test corpora.  For all languages except
Serbian, English and Turkish, we trained the language models using the
corresponding wikipedia dump files\footnote{Latest Wikipedia dump
files are freely available at \url{http://dumps.wikimedia.org/} and
the text in the dump files can be extracted using WP2TXT
(\url{http://wp2txt.rubyforge.org/})}[Should we talk about tokenizer?].

Serbian shares common basis with Croation and Bosnian thus we trained
3 different language models using Wikipedia dump files of Serbian
together with these two languages and measured the perplexities on
Serbian test corpus.  We chose the language model that uses Croation
dump file since it achieved the lowest perplexity score and unknown
word ratio on Serbian test corpus.

To train statistical language model of English, we used Wall Street
Journal data (1987-1994) extracted from CSR-III Text \cite{csr3text}
(we excluded the test corpus) and for the Turkish language modelling
we used the web corpus that was collected from Turkish news and blog
sites \cite{sak2008turkish}.  

We set the unknown word threshold to 2 for all languages except
English.  English has relatively low unknown word therefore we set the
threshold to 20 instead of 2.[double check for multext east.]

\section{Appendix B}
\label{app:tags}
Test2 Appendix~\ref{app:tags} This might require lots of spaces since
some languages has 80 tags.
