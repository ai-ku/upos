%% \section{Co-occurrence Modeling}
%% \label{sec:code}

\appendix
%%\appendixsection{Algorithm}
%%\label{sec:algorithm}
%% [!! remove this part]
%% In this section, we briefly describe the components of our algorithm.
%% Section~\ref{sec:subcomp} presents the motive and the method for
%% computation of substitute vectors, our paradigmatic representations
%% for the word contexts.  We combine the substitute vectors with the
%% identity and features of the target word for part of speech induction
%% using the a co-occurrence modeling framework.

\appendixsection{Computation of Substitute Distributions}
\label{app:subcomp}

In this study, we predict the syntactic category of a word in a given
context based on its substitute distribution.  The sample space of the
substitute distribution is the vocabulary of the language model
including the unknown word tag \unk.  Note that the substitute
distribution is a function of the context only and is indifferent to
the target word.

%% The dimensions of the
%% substitute distribution represent words in the vocabulary of the
%% language model, and the entries in the substitute distribution
%% represent the probability of those words being used in the given
%% context.  

% how are the substitutes computed
It is best to use both the left and the right context when estimating
the probabilities for potential lexical substitutes.  For example, in
\emph{``He lived in San Francisco suburbs.''}, the token \emph{San}
would be difficult to guess from the left context but it is almost
certain looking at the right context.  We define $c_w$ as the $2n-1$
word window centered around the target word position: $w_{-n+1} \ldots
w_0 \ldots w_{n-1}$ ($n=4$ is the n-gram order we have used).  The
probability of a substitute word $w$ in a given context $c_w$ can be
estimated as:
\begin{eqnarray}
  \label{eq:lm1}P(w_0 = w | c_w) & \propto & P(w_{-n+1}\ldots w_0\ldots w_{n-1})\\
  \label{eq:lm2}& = & P(w_{-n+1})P(w_{-n+2}|w_{-n+1})\nonumber\\
  &&\ldots P(w_{n-1}|w_{-n+1}^{n-2})\\
  \label{eq:lm3}& \approx & P(w_0| w_{-n+1}^{-1})P(w_{1}|w_{-n+2}^0)\nonumber\\
  &&\ldots P(w_{n-1}|w_0^{n-2})
\end{eqnarray}
where $w_i^j$ represents the sequence of words $w_i w_{i+1} \ldots
w_{j}$.  In Equation \ref{eq:lm1}, $P(w|c_w)$ is proportional to
$P(w_{-n+1}\ldots w_0 \ldots w_{n-1})$ because the words of the
context are fixed.  Terms without $w_0$ are identical for each
substitute in Equation \ref{eq:lm2} therefore they have been dropped
in Equation \ref{eq:lm3}.  Finally, because of the Markov property of
n-gram language model, only the closest $n-1$ words are used in the
experiments.

Near the sentence boundaries the appropriate terms were truncated in
Equation \ref{eq:lm3}.  Specifically, at the beginning of the sentence
shorter n-gram contexts were used and at the end of the sentence terms
beyond the end-of-sentence token were dropped.

To obtain a discrete representation of the context, the
random-substitutes algorithm pairs each word token with a substitute
sampled from the pre-computed substitute distribution generated from
the word token's context and then word ($W$) -- random-substitute
($S$) pairs are fed to the S-CODE algotihm as input.

\appendixsection{The CODE and S-CODE Models}
\label{app:codethr}

In this section we review the unsupervised method that we use to model
co-occurrence statistics: the Co-occurrence Data Embedding
(CODE)\ \cite{globerson2007euclidean} method and its spherical
extension (S-CODE) introduced by \cite{maron2010sphere}.

Let $W$ and $C$ be two categorical variables with finite cardinalities
$|W|$ and $|C|$.  We observe a set of pairs $\{w_i, c_i\}_{i=1}^n$
drawn IID from the joint distribution of $W$ and $C$.  The basic idea
behind CODE and related methods is to represent (embed) each value of
$W$ and each value of $C$ as points in a common Euclidean space
$\mathbf{R}^d$ such that values that frequently co-occur lie close to
each other.  There are several ways to formalize the relationship
between the distances and co-occurrence statistics, in this paper we
use the following:
\begin{equation} \label{eq:probability}
p(w,c) = \frac{1}{Z} \bar{p}(w) \bar{p}(c) e^{-d^2_{w,c}}
\end{equation}
\noindent where $d^2_{w,c}$ is the squared distance between the
embeddings of $w$ and $c$, $\bar{p}(w)$ and $\bar{p}(c)$ are empirical
probabilities, and $Z=\sum_{w,c} \bar{p}(w) \bar{p}(c) e^{-d^2_{w,c}}$
is a normalization term.  If we use the notation $\phi_w$ for the
point corresponding to $w$ and $\psi_c$ for the point corresponding to
$c$ then $d^2_{w,c} = \|\phi_w-\psi_c\|^2$.  The log-likelihood of a
given embedding $\ell(\phi, \psi)$ can be expressed as:
\begin{eqnarray}
&&\ell(\phi, \psi) = \sum_{w,c} \bar{p}(w,c) \log p(w,c) \label{eq:likelihood} \\
&&= \sum_{w,c} \bar{p}(w,c) (-\log Z + \log \bar{p}(w)\bar{p}(c) - d^2_{w,c}) \nonumber \\
&&= -\log Z + \mathit{const} - \sum_{w,c} \bar{p}(w,c) d^2_{w,c} \nonumber
\end{eqnarray}
The likelihood is not convex in $\phi$ and $\psi$.  We use gradient
ascent to find an approximate solution for a set of $\phi_w$, $\psi_c$
that maximize the likelihood.  The gradient of the $d^2_{w,c}$ term
pulls neighbors closer in proportion to the empirical joint
probability:
\begin{equation}
\frac{\partial}{\partial\phi_w} \sum_{w,c} -\bar{p}(w,c) d^2_{w,c} =
\sum_y 2 \bar{p}(w,c) (\psi_c - \phi_w) \label{eq:attract}
\end{equation}
The gradient of the $Z$ term pushes neighbors apart in proportion to the
estimated joint probability:
\begin{equation}
\frac{\partial}{\partial\phi_x} (-\log Z) = \sum_y 2 p(w,c) (\phi_w -
\psi_c) \label{eq:repulse}
\end{equation}
Thus the net effect is to pull pairs together if their estimated
probability is less than the empirical probability and to push them
apart otherwise.  The gradients with respect to $\psi_c$ are similar.
S-CODE \cite{maron2010sphere} additionally restricts all $\phi_w$ and
$\psi_c$ to lie on the unit sphere.  With this restriction, $Z$ stays
around a fixed value during gradient ascent.  This allows S-CODE to
substitute an approximate constant $\tilde{Z}$ in gradient
calculations for the real $Z$ for computational efficiency.  In our
experiments, we used S-CODE with its sampling based stochastic
gradient ascent algorithm and smoothly decreasing learning rate.

\appendixsection{S-CODE with More than Two Variables}

In order to accommodate multiple feature types the S-CODE model in the
previous section needs to be extended to handle more than two
variables.  Globerson et. al \shortcite{globerson2007euclidean}
suggest the following likelihood function:

\begin{eqnarray}
&\ell(\phi,& \psi^{(1)}, \ldots, \psi^{(K)}) = \label{eq:multicode}
  \bar{p}(w,c) \log p(w,c) + \sum_i^K \sum_{w,f^{(i)}} \bar{p}(w,f^{(i)}) \log p(w,f^{(i)})
\end{eqnarray}

\noindent where $\bar{p}(w,c)$ is the empirical joint distribution of
context $C$ with $W$, $F^{(1)}, \ldots, F^{(K)}$ are extra $K$
different variables whose empirical joint distributions with $W$,
$\bar{p}(w,f^{(1)})\ldots\bar{p}(w,f^{(K)})$, are known.
Eq.~\ref{eq:multicode} then represents a set of CODE models
$p(w,f^{(k)})$ where each $F^{(k)}$ has an embedding $\psi_f^{(k)}$
but all models share the same $\phi_w$ embedding.

We adopt this likelihood function, let $W$ represent a word, $C$
represent a context (i.e., random substitute), and $F^{(1)}, \ldots,
F^{(K)}$ stand for morphological and orthographic features of the word
thus each co-occurrence is a (K+1)-tuple, $(W, C, F^{(1)}, \hdots
F^{(K)})$.  With this setup, the training procedure needs to change
little: instead of sampling a word ($w$) -- context ($c$), the word
($w$) -- context ($c$) -- features ($f_1,\hdots,f_K$) tuple is sampled
and input to the gradient ascent algorithm.  The gradient search
algorithm updates the embeddings according to $p(w,c)$ and
$p(w,f^{(i)})$ where $i=1\hdots k$ and no updates are performed
between $c$ and $f^{(i)}$s since they do not have any co-occurrence
statistics and $w$ is the only shared variable.

Tuples might have null values due to unobserved features.  For example
in the case of POS induction, the word ``\textbf{car}'' has no
morphological or orthographic features therefore all the elements of
the tuple have null value except the word type ($w$) and the context
($c$).  We do not perform any pull or push updates on embeddings
during the gradient search if the corresponding $f^{(k)}$ is
null\footnote{In the POS induction problem $w$ and $c$ represents the
  word type and context therefore they are always observed.}.

\appendixsection{Language Statistics}

This section explains the language model training and feature
extraction of each language that we apply our model in
Section~\ref{sec:multilang}.  

\paragraph{Statictical Language Modeling}For all languages except
Serbian, English and Turkish, we train the language models by using
the corresponding Wikipedia dump files\footnote{Latest Wikipedia dump
  files are freely available at \url{http://dumps.wikimedia.org/} and
  the text in the dump files can be extracted using WP2TXT
  (\url{http://wp2txt.rubyforge.org/})}.  Serbian shares a common
basis with Croatian and Bosnian therefore we trained 3 different
language models using Wikipedia dump files of Serbian together with
these two languages and measured the perplexities on the MULTEXT-East
Serbian corpus.  We chose the Croatian language model since it
achieved the lowest perplexity score and unknown word ratio on the
MULTEXT-East Serbian corpus.  To train the statistical language model
of English, we use Wall Street Journal data (1987-1994) extracted from
CSR-III Text \cite{csr3text} (excluding sections of the PTB) and for
the Turkish language modeling we use the web corpus collected from
Turkish news and blog sites \cite{sak2008turkish}.  

In order to reduce the unknown word ratio of resource poor languages
and to standardize the process we set the vocabulary threshold to 2
for all languages except English.  English has a relatively low
unknown word ratio therefore we set the threshold to 20 instead of 2.
Table~\ref{tab:lmstatistics} summarizes the language model related
statistics and scores that vary across the languages in terms of
quality and quantity.

\paragraph{Feature extraction}Morphological features of each language
are extracted using the training sections of the corresponding
MULTEXT-East and CoNLL-X corpora.  We don't use the language model
corpora to extract morphological features.  Number of morphological
feature of each language is presented in Table~\ref{tab:lmstatistics}.
We use the same set of orthographic features described in
Section~\ref{sec:feat} except we add an ``Only-Punctuation'' feature
to the languages of MULTEXT-East corpora.  The ``Only-Punctuation''
feature is generated when a token only consists of punctuation
characters.

\begin{table}[ht]
%  \small
  \caption{Summary of language model training and test corpora
    statistics for each language in the test set.  Last two column
    presents the number of induced suffix parts and word types with
    these suffix parts after the morfological feature extraction.}
  \begin{tabular}{@{ }l@{ }|@{ }l@{ }|@{ }c@{ }|@{ }c@{ }|c@{ }|@{ }c@{ }|@{ }c@{ }|@{ }c@{ }|@{ }c@{ }|}
  \hline
    & & \multicolumn{2}{@{ }c@{ }|}{Language Model} & \multicolumn{5}{@{ }c@{ }|}{Test set}\\    \hline
    & Language & Source & \specialcell{Word\\Count} & \specialcell{Word\\Count} & \specialcell{Perplexity\\(ppl)} & \specialcell{Unknown\\Word} & \specialcell{Suffix\\Parts} & \specialcell{Word Types\\with\\Suffix parts}\\ \cline{1-9}
    \multirow{1}{*}{\begin{sideways}\textbf{WSJ}\end{sideways}} 
    &English & News & 126,170,376 & 1,173,766 & 79.926 & 0.012 & 5575 & 19223 \\
    & & & && & & &\\\hline
    \multirow{8}{*}{\begin{sideways}\textbf{MULTEXT-East}\end{sideways}}
    &Bulgarian& Wikipedia & 32,511,616 & 101,173 & 655.202 & .0565 & 609 & 4209\\
    &Czech & Wikipedia & 59,698,049 & 100,368 & 1,069.67 & .0299 & 2787 & 12848\\
    &English & News & 126,170,376 & 118,424 & 265.246 & .0288 & 1251 & 4783\\
    &Estonian & Wikipedia & 14,513,571 & 94,898 & 871.765 & .0654 & 4448 & 13638\\
    &Hungarian & Wikipedia & 66,069,788 & 98,426 & 742.676 & .0449 & 5423 & 15995\\
    &Romanian & Wikipedia & 35680870 & 118,328 & 666.855 & .1074 & 2064 & 9445\\
    &Slovene & Wikipedia & 18,969,846 & 112,278 & 658.711 & .0389 & 2093 & 11834\\
    &Serbian & Wikipedia & 17,129,679 & 108,809 & 804.962 & .0580 & 2722 & 12476\\
    \hline % Conll06 data
    \multirow{10}{*}{\begin{sideways}\textbf{CoNLL-X Shared Task}\end{sideways}}
    &Bulgarian& Wikipedia & 32,511,616 & 190,217 & 538.972 & .0430 & 926 & 8225\\
    &Czech & Wikipedia & 59,698,049 & 1,249,408 & 1,233.95 &.0250 & 12443 & 85673\\
    &Danish & Wikipedia & 35,863,945 & 94,386 & 351.24 & .0393 & 3708 & 10897\\
    &Dutch & Wikipedia & 159,978,524 & 195,069 & 390.818 & .0476 & 5250 & 13407\\
    &German & Wikipedia & 437,777,863 & 699,610 & 680.036 & .0487 & 15219 & 45414\\
    &Portuguese & Wikipedia & 150,099,154 & 206,678 & 378.656 & .0861 & 5033 & 15721\\
    &Slovene & Wikipedia & 18,969,846 & 28,750 & 663.053 & .0414  & 1257 & 4781\\
    &Spanish & Wikipedia & 332,311,650 & 89,334 & 274.418 & .0424  & 2648 & 9316\\
    &Swedish & Wikipedia & 32,004,538 & 191,467 & 1,233.95 & .0250  & 2897 & 12725\\
    &Turkish & Web & 491,195,991 & 47,605 & 868.829 & .0508  & 5651 & 14227\\
    \hline
  \end{tabular}
  \label{tab:lmstatistics}
\end{table}

%% \begin{table}[ht]
%%   %\tiny  
%%   \caption{Summary of language model training and test corpora
%%   statistics for each language in the test set.} 
%%   \begin{tabular}{@{ }l@{ }|@{ }l@{ }|@{ }c@{ }|@{ }c@{ }|@{ }c@{ }|c@{ }|@{ }c@{ }|@{ }c@{ }|@{ }c@{ }|}
%%   \hline
%%     & & \multicolumn{3}{c|}{Language Model} & \multicolumn{4}{c|}{Test set}\\    \hline
%%     & Language & Source & \specialcell{Sentence\\Count} & \specialcell{Word\\Count} & \specialcell{Sentence\\Count} & \specialcell{Word\\Count} & \specialcell{Perplexity\\(ppl)} & \specialcell{Unknown\\Word} \\ \cline{1-9}
%%     \multirow{1}{*}{\begin{sideways}\textbf{WSJ}\end{sideways}} 
%%     &English & News & 5,187,874 & 126,170,376 & 49,208 & 1,173,766 & 79.926 & 0.012\\
%%     & & & & && & &\\\hline
%%     \multirow{8}{*}{\begin{sideways}\textbf{MULTEXT-East}\end{sideways}}
%%     &Bulgarian& Wikipedia &1,596,399 & 32,511,616  & 6,682 & 101,173 & 655.202 & .0565\\
%%     &Czech & Wikipedia &3,059,678 & 59,698,049 & 6,752 & 100,368 & 1,069.67 & .0299\\
%%     &English & News & 5,187,874 & 126,170,376 & 6,737 & 118,424 & 265.246 & .0288\\
%%     &Estonian & Wikipedia &833,677 & 14,513,571 & 6,478 & 94,898 & 871.765 & .0654\\
%%     &Hungarian & Wikipedia &3,250,267& 66,069,788 & 6,768 & 98,426 & 742.676 & .0449\\
%%     &Romanian & Wikipedia &3,250,267&66,069,788  & 6,520 & 118,328 & 666.855 & .1074\\
%%     &Slovene & Wikipedia & 899,329&18,969,846 & 6,689 & 112,278 & 658.711 & .0389\\
%%     &Serbian & Wikipedia & 782,278 & 17,129,679 & 6,677 & 108,809 & 804.962 & .0580\\
%%     \hline % Conll06 data
%%     \multirow{10}{*}{\begin{sideways}\textbf{CoNLL-X Shared Task}\end{sideways}}
%%     &Bulgarian& Wikipedia &1,596,399 & 32,511,616  & 12,823 & 190,217 & 538.972 & .0430\\
%%     &Czech & Wikipedia &3,059,678 & 59,698,049 & 72,703 & 1,249,408 & 1,233.95 &.0250\\
%%     &Danish & Wikipedia &1,672,003 & 35,863,945 & 5,190 & 94,386 & 351.24 & .0393\\
%%     &Dutch & Wikipedia &8,266,922 & 159,978,524 & 13,349 & 195,069 & 390.818 & .0476\\
%%     &German & Wikipedia &22,454,543&437,777,863 & 39,216 & 699,610 & 680.036 & .0487\\
%%     &Portuguese & Wikipedia & 5,706,037 & 150,099,154 & 9071 & 206,678 & 378.656 & .0861\\
%%     &Slovene & Wikipedia & 899,329 & 18,969,846 & 1,534 & 28,750 & 663.053 & .0414\\
%%     &Spanish & Wikipedia &11,534,351 & 332,311,650& 3,306 & 89,334 & 274.418 & .0424\\
%%     &Swedish & Wikipedia &1,953,794 & 32,004,538& 11,042 & 191,467 & 1,233.95 & .0250\\
%%     &Turkish & Web &39,595,781 & 491,195,991& 4,997 & 47,605 & 868.829 & .0508\\
%%     \hline
%%   \end{tabular}
%%   \label{tab:lmstatistics}
%% \end{table}

%% S-CODE handles two variables, whereas underlying syntactic categories
%% can be captured by more than two different variables such as
%% contextual, morphologic and ortographic features.  S-CODE can be
%% extented to handle more than two variables in a way similar to the
%% multi variable extension of CODE \cite{globerson2007euclidean} with
%% the unit sphere restriction.  The log-likelihood at
%% Equation~\ref{eq:likelihood} can be redefined for $n+1$ different
%% categorical variables $X$, $Y_i$, $\hdots$ and $Y_n$ with finite
%% cardinalities $|X|$, $|Y_1|$, $\hdots$ and $|Y_n|$, respectively, as:
%% \begin{eqnarray}
%% &&\ell(\phi, \psi_1,\hdots,\psi_n) = \sum_{i=1}^n\sum_{x,y_i} \bar{p}(x,y_i) \log p(x,y_i) \label{eq:multiscode} \\
%% &&= \sum_{i=1}^n\sum_{x,y_i} \bar{p}(x,y_i) (-\log Z_i + \log \bar{p}(x)\bar{p}(y_i) - d^2_{x,y_i}) \nonumber \\
%% &&=-\sum_{i=1}^n(\log Z_i + \mathit{const}_i - \sum_{x,y_i} \bar{p}(x,y_i) d^2_{x,y_i}) \nonumber
%% \end{eqnarray}
%% where $\psi_i$ is the embedding of $y_i \in Y_i$ and $Z_i$ is the
%% normalization term of $p(x,y_i)$.  Thus the model is able to jointly
%% learn the embeddings when the pairwise co-occurence statistics,
%% $\bar{p}(x,y_i)$, are available for all $i$.

%% One problem with these setting is, not every $(x,y_i)$ pair is
%% observed in the data.  For example, the stem word ``\textbf{car}''
%% doesn't have any morphological feature, thus its morphological feature
%% is represented by a null value, ``X''.  However setting the unobserved
%% features to ``X'' leads to pulling the words with unobserved features
%% together even they are from different clusters or pushing the ones
%% with observed features apart even they are from same clusters.  To
%% solve this, during the gradient search we don't perform any pull or
%% push updates on embeddings if the value of $y_i$ is set to null.
