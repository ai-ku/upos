\section{Experiments}
\label{sec:multilang}
%%Multiple laguage section
\begin{table}[ht]
%% Table notation info
%% PYP is \ddagger
%% BMMM is \start
%% Clark is \dagger
%% kmeans is ^k
%% 
%  \tiny
  \small
  \caption{The \mto\ and \vm\ scores on 19 corpora in 15 languages
    together with the number of types and tags in gold--set which
    equals to number of induced clusters in all languages.  Best
    published results are from
    $^\ddagger$\protect\cite{blunsom-cohn:2011:ACL-HLT2011},
    $^\star$\protect\cite{christodoulopoulos-goldwater-steedman:2011:EMNLP}
    and $^\dagger$\protect\cite{Clark:2003:CDM:1067807.1067817}. Bold
    results represent the best \mto\ and \vm\ accuracies of the
    corresponding language.  MULTEXT-East corpora do not tag the
    punctuation marks, thus we add an extra tag for punctuation and
    represent it with ``+1''.}
  \begin{flushleft}
%  \begin{tabular}{|l|l|l|l|p{2cm}|p{2cm}|p{2cm}|p{2cm}|p{2cm}|}
  \begin{tabular}{|@{ }l@{ }|@{ }l@{ }|@{ }l@{ }|@{ }l@{ }|@{ }l@{ }|@{ }l@{ }|@{ }l@{ }|@{ }l@{ }|@{ }l@{ }|}
        \hline
        & Language   & Types   & Tags & \specialcell{Best\\Published}            & \specialcell{Syntagmatic\\Bigram} &DIS           & DIS+O            & DIS+O+M     \\ \hline % Multext data
        \multirow{1}{*}{\begin{sideways}\textbf{WSJ}\end{sideways}} 
        & English    & 49,190  & 45 & .775 / .697$^\ddagger$  &.7314 / .6558 & .7680 / .6822 & .7830 / .7026 & \textbf{.8004 / .7160}           \\
        & & & & & & & &\\ \hline
        \multirow{8}{*}{\begin{sideways}\textbf{MULTEXT-East}\end{sideways}}
        & Bulgarian    & 16,352  & 12+1 & .665 / \textbf{.556}$^\star$    & .6732 / .4119 & .6883 / .5291 & \textbf{.7039} / .5496     & .6754 / .5246 \\
        & Czech      & 19,115  & 12+1 & .642 / \textbf{.539}$^\star$    & .6269 / .4586 & .6781 / .4829 & .6742 / .4854     & \textbf{.6977} / .5042 \\
        & English    & 9,773   & 12+1 & .733 / .633$^\star$    & .7690 /. 6131 & .8229 / .6610 & .8282 / .6719     & \textbf{.8343 / .6787} \\
        & Estonian   & 17,845  & 12+1 & .644 / \textbf{.533}$^\star$    & .6089 / .4119 & .6555 / .4437 & \textbf{.6634} / .4606     & .6526 / .4418 \\
        & Hungarian  & 20,321  & 11+1 & .682 / \textbf{.548}$^\star$    & .6181 / .4514 & .6914 / .5046 & .7052 / .5244     & \textbf{.7287} / .5444 \\
        & Romanian   & 15,189  & 14+1 & .611 / .523$^\star$    & .6565 / .5202 & .6469 / .5012 & \textbf{.6675 / .5269}     & .6488 / .5251 \\
        & Slovene    & 17,871  & 12+1 & .679 / \textbf{.567}$^\star$    & .6772 / .5044 & .6873 / .4845 & \textbf{.6892} / .4901     & .6833 / .4941 \\
        & Serbian    & 18,095  & 12+1 & .641 / \textbf{.510}$\dagger$        & .6267 / .4510 & .6240 / .4479 & .6303 / .4554     & \textbf{.6368} / .4650 \\
        \hline % Conll06 data
        \multirow{10}{*}{\begin{sideways}\textbf{CoNLL-X Shared Task}\end{sideways}}
%        & Arabic     & 12,915  & 20 & -   & & - & -               & -  \\
        & Bulgarian  & 32,439  & 54   & .704 / \textbf{.596}$\dagger$        & .6972 / .5532 & \textbf{.7399} / .5824  & .7391 / .5856     & .7207 / .5673 \\
%        & Chinese    & 40,562  & 15 & -   & & & - & -           \\
        & Czech      & 130,208 & 12   & .701$^\ddagger$ / .484$^\star$           & .6944 / .5036 & .6764 / .4867  & \textbf{.7149 / .5330}     & .6903 / .5227 \\
        & Danish     & 18,356  & 25   & \textbf{.761}$^\ddagger$ / .591$^\star$           & .6757 / .5290 & .7214 / .5559  & .7520 / .5927     & .7482 / \textbf{.5958} \\
        & Dutch      & 28,393  & 13   & .711$^\ddagger$ / .547$^\start$ & .6703 / .5205 & .7014 / .5405  & \textbf{.7393 / .5980}     & .7228 / .5925 \\
        & German     & 72,326  & 54   & .744$^\star$ / .630$\dagger$& .7525 / .6285 & .7637 / .6314  & \textbf{.7735 / .6554}     & .7529 / .6403 \\
%        & Japanese   & 3,231   & 80 & -   & - & - & -               & -           \\
        & Portuguese & 28,931  & 22   & .785$^\ddagger$ / .639$^\star$           & .7031 / .5617 & .7381 / .5770  & .7907 / .6317     & \textbf{.7948 / .6405} \\
        & Slovene    & 7,128   & 29   & .642$^\star$ / \textbf{.539}$\dagger$    & .6384 / .4976 & .6503 / .4925  & .6555 / .5036     & \textbf{.6572} / .5023 \\
        & Spanish    & 16,458  & 47   & \textbf{.788}$^\ddagger$ / .632$^\star$          & .7086 / .5844 & .7492 / .6083  & .7718 / \textbf{.6372}     & .7627 / .6331 \\
        & Swedish    & 20,057  & 41   & .682 / \textbf{.589}$\dagger$        & .6721 / .5558 & .6931 / .5654  & \textbf{.6946} / .5721     & .6649 / .5613 \\
        & Turkish    & 17,563  & 30   & .628 / .408$^\star$        & .6069 / .3551 & .6228 / .3804  & .6348 / .4109     & \textbf{.6500 / .4246} \\ \hline % extra data from cristo2011 paper
%        & French     & 49,964  & 23 & - & & - & -               & -        \\
%        & A.Greek    & 15,194  & 15 & - & & - & -               & -        \\ \hline
    \end{tabular}
  \end{flushleft}
  \label{tab:multiresults}
\end{table}
 

Following Christodoulopoulos et
al. \shortcite{christodoulopoulos-goldwater-steedman:2011:EMNLP}, we
evaluate our model on the Penn Treebank 1M word WSJ English corpus
together with 8 languages from MULTEXT-East (Bulgarian, Czech,
English, Estonian, Hungarian, Romanian, Slovene and Serbian)
\cite{citeulike:5820223} and 10 languages from the CoNLL-X shared task
(Bulgarian, Czech, Danish, Dutch, German, Portuguese, Slovene,
Spanish, Swedish and Turkish)
\cite{Buchholz:2006:CST:1596276.1596305}.  During the testing we only
use the training section of MULTEXT-East and CONLL-X languages
\cite{Lee:2010:STU:1870658.1870741}.

Languages of MULTEXT-East corpora do not tag the punctuation marks,
thus we add an extra tag for punctuation to the tag-set of these
languages.  The number of word types and clusters of each language are
summarized in Table~\ref{tab:multiresults}.

\subsection{Substitute Vectors}
To calculate top 100 substitutes of each position, we trained a 4-gram
language model with the corresponding training corpora of each
language\footnote{See Appendix~\ref{app:lm} for the further details on
  the statistics of language model.}  as described in
Section~\ref{sec:subthr}.  The word count of Wikipedia dump files vary
across the languages, in order to reduce the unknown word ratio of
resource poor languages and to standardize the process we set the
unknown word threshold to 2 for all languages except English.  For
English, we set the unknown word threshold to 20.
%%Language model statistics are detailed in Appendix~\ref{app:lm}.

\subsection{Features}
We use the same set of orthographic features described in
Section~\ref{sec:feat} except we add an extra ``Only-Punctuation''
feature to the languages of MULTEXT-East corpora.  The
``Only-Punctuation'' is generated when a token contains only
punctuations.  

Morphological features of each language is extracted by the method
described in Section~\ref{sec:feat}.  The number of suffix types are
summarized in Appendix~\ref{morpho}.

\subsection{Results}
We perform experiments with a range of languages and three different
feature setups to establish both the robustness of our model across
languages and to observe the effects of different features.  For each
language we report results: (1) without features (uPos), (2) with
orthographic features (uPos+O) and (3) with both orthographic and
morphological features (uPos+O+M).  In accord with
Section~\ref{sec:wordsub}, we use the 25 dimensional sphere with 64
substitutes for all languages.  As a baseline model we chose the
bi-gram version of S-Code described in Section~\ref{sec:bigram} which
is a very strong baseline compared to the ones used in
\cite{christodoulopoulos-goldwater-steedman:2011:EMNLP}.
Table~\ref{tab:multiresults} summarizes the \mto and \vm scores of our
models together with the bi-gram baseline and the best published
accuracies on each language corpus.

uPos significantly outperformed the bi-gram baseline in both \mto and
\vm scores on 14 languages while bi-gram model performed better on
Romanian, Serbian and Czech (CONLL-X).  uPos+O+M has the
state-of-the-art \mto and \vm accuracy on WSJ English corpus.  uPos+O
and uPos+O+M achieved the highest \mto scores on all languages of
MULTEXT-East corpora while scoring the highest \vm accuracies on
English and Romanian.  On the CoNLL-X languages Pos+O and uPos+M
models perform better than the best published \mto score on 7
languages (Czech, Dutch, German, Portuguese, Slovene, Swedish and
Turkish).  Similarly, these models achieved the top \vm scores on 7
languages (Czech, Danish, Dutch, German, Portuguese, Spanish and
Turkish).  uPos achieves the best published \mto score on CONLL-X
Bulgarian corpora.

