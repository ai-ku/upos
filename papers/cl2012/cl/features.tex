\subsection{Morphological and Orthographic Features}
\label{sec:feat}

Clark \shortcite{Clark:2003:CDM:1067807.1067817} demonstrates that
using morphological and orthographic features significantly improves
part of speech induction with an HMM based model.
Section~\ref{sec:related} describes a number of other approaches that
show similar improvements.  This section describes one way to
integrate additional features to the random-substitute model.

In order to accommodate multiple feature types the CODE model in
Appendix~\ref{sec:codethr} needs to be extended to handle more than
two variables.  Globerson et al. \shortcite{globerson2007euclidean}
suggest the following likelihood function:

\begin{eqnarray}
&\ell(\phi,& \psi^{(1)}, \ldots, \psi^{(K)}) = \label{eq:multiscode} \sum_i^K w_i \sum_{x,y^{(i)}} \bar{p}(x,y^{(i)}) \log p(x,y^{(i)})
\end{eqnarray}

\noindent where $Y^{(1)}, \ldots, Y^{(K)}$ are $K$ different variables
whose empirical joint distributions with $X$,
$\bar{p}(x,y^{(1)})\ldots\bar{p}(x,y^{(K)})$, are known.
Eq.~\ref{eq:multiscode} then represents a set of CODE models
$p(x,y^{(k)})$ where each $Y^{(k)}$ has an embedding $\psi_y^{(k)}$
but all models share the same $\phi_x$ embedding.  The weights $w_k$
reflect the relative importance of each $Y^{(k)}$ and all embeddings
are mapped to unit-sphere.

We adopt this likelihood function, set all $w_k=1$, let $X$ represent
a word, $Y^{(1)}$ represent a random substitute, and $Y^{(2)}, \ldots,
Y^{(K)}$ stand for morphological and orthographic features of the word
thus each word is a (K+1)-tuple, $(X, Y^{(1)}, \hdots Y^{(K)})$.  With
this setup, the training procedure needs to change little: instead of
sampling a word -- random-substitute pair, the word --
random-substitute -- features tuple is sampled and input to the
gradient ascent algorithm.  The gradient search algorithm updates the
embeddings according to $p(x,y^{(i)})$ where $i=1\hdots k$ and no
updates are performed between $y^{(i)}$s since they do not have any
co-occurrence statistics and $x$ is the shared variable.

Word tuples might have null values due to the unobserved features.
For example, the word ``\textbf{car}'' has no morphological or
orthographic features therefore all the elements of the tuple have
null value except the word type ($X$) and the random-substitute
($Y^{(1)}$).  We do not perform any pull or push updates on embeddings
during the gradient search if the corresponding $y^{(k)}$ is
null\footnote{$X$ and $Y^{(1)}$ represents the word type and
  random-substitute therefore they are always observed.}.

%% One problem with this setup is that unobserved features misguide the
%% gradient search algorithm and lead to a suboptimal convergence point.
%% For example, ``\textbf{car}'' and ``\textbf{red}'' belong to the
%% ``Noun'' and ``Adjective'' clusters, respectivly, and neither of them
%% have a morphological feature, thus their morphological features are
%% represented by a null value, ``X''.  However setting the unobserved
%% features of words from different clusters to ``X'' leads to a false
%% similarity between these words.  To solve this problem, 

The orthographic features we used are similar to the ones in
\cite{bergkirkpatrick-EtAl:2010:NAACLHLT} with small modifications:

\begin{itemize}
\item Initial-Capital: this feature is generated for capitalized words
  with the exception of sentence initial words.
\item Number: this feature is generated when the token starts with a
  digit.
\item Contains-Hyphen: this feature is generated for lowercase words
  with an internal hyphen.
\item Initial-Apostrophe: this feature is generated for tokens that
  start with an apostrophe.
\end{itemize}

We generated morphological features using the unsupervised algorithm
Morfessor \cite{creutz05}.  Morfessor was trained on the WSJ section
of the Penn Treebank using default settings, and a perplexity
threshold of 1.  In our model, a word type consists of two parts: a
stem and a suffix part.  The suffix part is used as the morphological
feature thus each word type has only one morphological
feature\footnote{We extracted the stem part by concatenating the
  splits until including the first ``STM'' labeled split and the
  suffix part by concatenating rest of the splits.}.  The program
induced 5575 suffix types that are present in a total of 19223 word
types.

%% These suffixes were input to S-CODE as morphological features whenever
%% the associated word types were sampled.  In order to incorporate
%% morphological and orthographic features into
%% S-CODE we modified its input.  For each word -- random-substitute pair
%% generated as in the previous section, we added word -- feature pairs
%% to the input for each morphological and orthographic feature of the
%% word.  Words on average have 0.25 features associated with them.
%% This increased the number of pairs input to S-CODE from 14.1
%% million (12 substitutes per word) to 17.7 million (additional 0.25
%% features on average for each of the 14.1 million words).

Using the training settings of the previous section, the addition of
morphological and orthographic features increased the many-to-one
score of the random-substitute model to \ftmto\ and V-measure to
\ftvm.  Both these results improve the state-of-the-art in
part of speech induction significantly as seen in
Table~\ref{tab:results}.

%% \subsection{Summary}
%% \label{sec:expsum}

%% We presented methods to determine the best usage of substitute vectors
%% within the CODE framework and performed sensitivity analysis on the
%% model parameters to not only show the robustness but also to decide
%% the best configuration of S-CODE in modeling the co-occurrence of
%% words with their contexts.  To use the substitute vectors as
%% co-occurrences we discretized the substitute vectors using two
%% methods.  The first method (Section~\ref{sec:rpart}) selected 64K
%% random substitute vectors as the center of 64K partitions and then
%% assigned rest of the substitute vectors to the closest partition.  As
%% a result each word represented as a word -- partition-id pair which we
%% input to S-CODE. The second method (Section~\ref{sec:wordsub}) sampled
%% random substitutes from the substitute vectors using the fact that the
%% substitute vectors are probability distributions.  We fed the word --
%% random-substitute pairs into S-CODE.  Both methods significantly
%% outperform the syntagmatic bigram model (Section~\ref{sec:bigram}) on
%% the PTB.  Finally, Section~\ref{sec:feat} showed that adding
%% morphological and orthographic features improved the accuracy and we
%% achieved the state-of-the-art \mto\ and \vm\ accuracies on the PTB.

%% In the next section we extend our experiments to include more
%% languages to demonstrate the robustness of paradigmatic approach on
%% languages with different characteristics.
