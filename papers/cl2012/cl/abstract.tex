%% TODO: can we mention type/token here without confusing everyone?

\begin{abstract}

We investigate paradigmatic representations of word context in the
domain of unsupervised part of speech induction.  Paradigmatic
representations of word context are based on potential substitutes of
a word in contrast to syntagmatic representations based on its
neighbors.  In preliminary experiments we find that clustering word
contexts (without considering target words) gives limited results and
conclude that it is important to consider the co-occurrence of a word
and its context for part-of-speech induction.  We model the joint
probability of words and their contexts (as represented by potential
substitutes) using the S-CODE framework \cite{maron2010sphere}.
S-CODE maps target words, their potential substitutes and other
features to high dimensional Euclidean vectors.  These vectors
aggregate into clusters that largely match the traditional
part-of-speech boundaries and give state-of-the-art results in
unsupervised part-of-speech induction, including 80\% many-to-one
accuracy on the Penn Treebank and statistically significant
improvements over best published results on 17 out of 19 corpora in 15
languages.

\end{abstract}
